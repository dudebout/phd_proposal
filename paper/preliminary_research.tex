The following presentation of the preliminary research was first developed in~\cite{dudebout_shamma:2012}.

\section{Empirical-evidence Equilibria}
\label{sec:empirical_evidence_equilibria}

\subsection{Single-agent Setup}

Consider a discrete-time dynamical system governed by
\begin{equation}
\label{eq:private_dynamic}
x\nxt \drawn  f \of{x, a, s},
\end{equation}
where \(x\) is a state, \(a\) is an action, and \(s\) is a signal.
Variables~\(x\), \(a\), and \(s\) take values in finite sets \(\cX\), \(\cA\), and \(\cS\), respectively.
The agent picks the action \(a\).
Nature determines the signal \(s\) according to
\begin{subequations}
\label{eq:nature_system}
\begin{align}
\label{eq:nature_dynamic}
w\nxt &\drawn n \of{w, x, a}, \\
\label{eq:nature_signal}
s &\drawn \nu \of{w},
\end{align}
\end{subequations}
where \(w\) is a state of Nature evolving in the finite state space~\(\cW\).
The agent observes \(s\) but not \(w\).
Denote by~\(\rbN\) the dynamical system described by~\cref{eq:private_dynamic,eq:nature_system}.

Define the agent's observation by \(o = \tuple{x, a, s}\) and the actual realization of the system by \(r = \tuple{w, x, a, s}\).
At time~\(t\), the agent's private history is \(p\T = \tuple{\sseqcom{o}{\tm}{0}{1}{t}}\) and the true history is \(h\T = \tuple{\sseqcom{r}{\tm}{0}{1}{t}}\).
Denote by \(\cP\) the set of finite private histories.
A strategy~\(\sigma : \cP \to \distribover{\cA}\) is a mapping from private histories to a distribution over the actions.

At each time step, the agent receives a payoff according to the utility function \(u: \cX \times \cA \times \cS \to \bR\).
For a given infinite private history the agent receives the sum of discounted payoffs~\(\sum\idxfromto{t}{0}{\infty} \delta^t u \of{x\T, a\T, s\T}\), where \(\delta \in \interval[oo]{0}{1}\) is a discount factor.
The agent wants to find a strategy maximizing its expected sum of discounted payoffs \(U\params{\rbN, \sigma}\of{x\tm{0}} = \expectof[\rbN, \sigma]{\sum\idxfromto{t}{0}{\infty} \delta^t u \of{x\T, a\T, s\T}}\).

When the agent knows~\cref{eq:private_dynamic,eq:nature_system}, it is facing a \ac{pomdp}.
A natural solution concept for this type of problems is an optimal policy for the \ac{pomdp}.
The agent computes an optimal policy making use of beliefs, which are probability distributions over the true history.
Beliefs are obtained from the private histories~\(p\T\), the signaling structure~\cref{eq:nature_system}, and the application of Bayes's rule.
Belief computation is intractable because every observation increases the size of the belief space.

When the agent knows~\cref{eq:private_dynamic} but does not know~\cref{eq:nature_system} it can still implement an optimal policy for the \ac{pomdp}.
However it cannot compute such an optimal policy anymore.
In such a setting, a less constraining solution concept is required.
Empirical-evidence optimality is one such solution concept that relies on the notion of statistical consistency.

The following section presents the simplest notion of statistical consistency, depth-\(k\) consistency.

\subsection{Depth-\texorpdfstring{\(k\)}{k} Consistency}
\label{sec:depth-k_consistency}

Consider \(c\), an \(\cS\)-valued ergodic process.
For \(k\) in \(\bN\), its depth-\(k\) characteristic \(\chi\K\) is the long-run distribution of the strings of length \(k + 1\).
For \(d\) in~\(\cS^{k + 1}\)
\begin{equation}
\label{eq:depth-k_characteristic}
\chi\K\elmt{d} = \limfty{t} \probaof{\tuple{\seqqcom{c}{\tm}{t - k}{t - 1}{t}} = d}.
\end{equation}
Two processes with the same depth-\(k\) characteristic are called depth-\(k\) consistent.

The signal observed by the agent is one such \(\cS\)-valued process.
Consider another \(\cS\)-valued process described by
\begin{subequations}
\label{eq:exogenous_mockup_system}
\begin{align}
\label{eq:exogenous_mockup_model}
z\nxt &= m\K \of{z, s}, \\
\label{eq:exogenous_mockup_predictor}
s &\drawn \mu \of{z},
\end{align}
\end{subequations}
where \(z\) is a state in \(\cS\pow{k}\) and \(m\K\) is the length-\(k\)--memory function defined by
\[
m\K \of{\tuple{\seqqcom{s}{\tm}{t - k}{t - 2}{t - 1}}, s\T} = \tuple{\seqqcom{s}{\tm}{t - k + 1}{t - 1}{t}}.
\]
Under some technical assumptions, described in~\cref{sec:empirical_evidence_optimality}, the observed signal and the Markov chain described by~\cref{eq:exogenous_mockup_system} are ergodic processes.
Furthermore, the Markov chain is depth-\(k\) consistent with the true signal when the following equality holds:
\[
\mu \of{z} \elmt{s} = \limfty{t}\probacond[\rbN, \sigma]{s\T = s}{\tuple{\seqqcom{s}{\tm}{t - k}{t - 2}{t - 1}} = z}.
\]

Denote by \(\rbM\K\) the dynamical system described by~\cref{eq:private_dynamic,eq:exogenous_mockup_system}.
The system \(\rbM\K\) induces an \ac{mdp} with state~\(\tuple{x, z}\), action~\(a\), strategy~\(\sigmahat \from \cX \times \cZ \to \cA\), and the objective function~\(U\params{\rbM\K, \sigmahat}\of{x\tm{0}, z\tm{0}} = \expectof[\rbM\K, \sigmahat]{\sum\idxfromto{t}{0}{\infty} \delta^t u \of{x\T, a\T, s\T}}\).
A strategy \(\sigmahat\) for the \ac{mdp} can be implemented in the real system by building \(z\) with~\cref{eq:exogenous_mockup_model}.
From now on, no distinction will be made between a strategy for the \ac{mdp} \(\sigmahat\) and its associated strategy built with~\cref{eq:exogenous_mockup_model}.
Both strategies will be denoted~\(\sigma\).

Consider the following iterative process.
The agent implements an initial strategy \(\sigma\tm{0}\).
It formulates a depth-\(k\) consistent model \(\mu\tm{0}\) of Nature's dynamic.
Then, it computes an optimal strategy \(\sigma\tm{1}\) for the \ac{mdp} induced by this model~\(\mu\tm{0}\).
Upon implementation of this new strategy, the model \(\mu\tm{0}\) may lose the requisite statistical consistency.
Therefore, the agent formulates a revised depth-\(k\) consistent model \(\mu\tm{1}\) and the process repeats.
A fixed point of this iterative process is one way to define a solution to this problem.
A strategy is a solution if it is optimal with respect to the model it induces.
Note that such a strategy is not a solution to the \ac{pomdp}.

Using that model to design a strategy is equivalent to the agent making an assumption about the system.
For example, when the agent uses a depth-\(k\) consistent model, it assumes the signal is generated exogenously, \ie, not impacted by \(x\) or \(a\).
This assumption might seem restrictive.
However, note that the repeated-modeling and optimization phases create a feedback loop.
Therefore, a model satisfying the consistency condition is exogenous but captures characteristics of Nature's dynamic.

The following section extends beyond the notion of depth-\(k\) consistency.

\subsection{Empirical-evidence Optimality}
\label{sec:empirical_evidence_optimality}

The agent assumes that a Markov chain, with state \(z\) from a finite set \(\cZ\), generates the signal \(s\) and that it can construct~\(z\) from its observations as follows:
\begin{subequations}
\label{eq:mockup_system}
\begin{align}
\label{eq:mockup_model}
z\nxt &\drawn m \of{z, x, a, s}, \\
\label{eq:mockup_predictor}
s &\drawn \mu \of{z}.
\end{align}
\end{subequations}
The model \(m\) represents the assumption the agent makes about the system.
The predictor~\(\mu\) is the set of parameters the agent adjusts to be consistent with its observations.
The pair~\(\mmu\) is called a mockup.

In this setup, depth-\(k\) consistency is replaced with the following definition.
\begin{definition}
\label{def:consistency}
Let \(\sigma\) be a strategy and \(\mmu\) be a mockup.
Predictor \(\mu\) is \(\sigmam\) consistent with \(\rbN\) if
\[
\mu \of{z} \elmt{s} = \limfty{t}\probacond[\rbN, \sigma]{s\Tp = s}{z\T = z}.
\]
\end{definition}

The notion of optimality used is the following.
\begin{definition}
Let \(\sigma\) be a strategy, \(\mmu\) be a mockup, and \(\epsilon\) be a positive number.
Strategy \(\sigma\) is \(\mum\) optimal if it is optimal for the \ac{mdp} induced by~\(\rbM\).
Strategy \(\sigma\) is \(\epsilonmum\)~optimal if it is \(\epsilon\) optimal for the \ac{mdp} induced by~\(\rbM\).
\end{definition}

Having defined consistency and optimality the definition of an \ac{eeo} follows.
\begin{definition}
Let \(\sigma\) be a strategy, \(\mmu\) be a mockup, and \(\epsilon\) be a positive number.
The pair \(\sigmamu\) is an \(m\) \ac{eeo} if the following two conditions hold:
\begin{enumerate}
\item Strategy \(\sigma\) is \(\mum\) optimal.
\item Predictor \(\mu\) is \(\sigmam\) consistent with \(\rbN\).
\end{enumerate}
The pair \(\sigmamu\) is an \(\epsilonm\) \ac{eeo} if the following two conditions hold:
\begin{enumerate}
\item Strategy \(\sigma\) is \(\epsilonmum\) optimal.
\item Predictor \(\mu\) is \(\sigmam\) consistent with \(\rbN\).
\end{enumerate}
\end{definition}

A little care must be taken to make \(\mu\) in~\cref{def:consistency} well defined.
Insuring the following assumption is verified guarantees it.
\begin{assumption}
\label{ass:ergodicity}
Let \(\sigma\) be a strategy, and \(T\strat{\sigma}\) be the Markov chain with state \(X = \tuple{w, x, z}\) induced by \(\rbN\) and \(\sigma\), \(X\nxt \drawn T\strat{\sigma} X\).
The Markov chain \(T\strat{\sigma}\) is ergodic.
\end{assumption}

\Cref{ass:ergodicity} insures that \(T\strat{\sigma}\) has a unique stationary distribution \(\pi\strat{\sigma}\) such that
\[
\limfty{t}\probacond[\rbN, \sigma]{s\Tp = s}{z\T = z} = \probacond[\pi\strat{\sigma}]{s}{z}.
\]
Furthermore, \cref{ass:ergodicity} guarantees that \(\pi\strat{\sigma}\) has full support, meaning that for all \(w\) in \(\cW\), \(x\) in \(\cX\), and \(z\) in \(\cZ\), \(\pi\strat{\sigma}\elmt{w,x,z}\) is positive.
This guarantees that \(\mu\) in~\cref{def:consistency} is well defined for all \(z\) and \(s\) as follows:
\begin{align*}
\mu \of{z} \elmt{s}
&= \limfty{t}\probacond[\rbN, \sigma]{s\Tp = s}{z\T = z} \\
&= \probacond[\pi\strat{\sigma}]{s}{z} \\
&= \sum\idxin{w}{\cW} \probacond[\pi\strat{\sigma}]{s}{z, w} \cdot \probacond[\pi\strat{\sigma}]{w}{z} \\
&= \sum\idxin{w}{\cW} \probacond[\pi\strat{\sigma}]{s}{w} \cdot \frac{\probaof[\pi\strat{\sigma}]{w, z}}{\probaof[\pi\strat{\sigma}]{z}} \\
&= \sum\idxin{w}{\cW} \nu\of{w}\elmt{s} \cdot \frac{\sum\idxin{x}{\cX} \pi\strat{\sigma}\elmt{w,x,z}}{\sum\idxin{w'}{\cW}\sum\idxin{x}{\cX} \pi\strat{\sigma}\elmt{w',x,z}}
\end{align*}
Consistency yields a mapping associating to a strategy \(\sigma\) a unique predictor \(\sigmam\) consistent with \(\rbN\).
Note that \(\mu\) is a continuous function of \(\pi\strat{\sigma}\).

Similarly, a mapping associating to a predictor \(\mu\) a unique \(\epsilonm\)-optimal strategy can be defined.
Denote by \(\rbM\) the dynamical system described by \cref{eq:private_dynamic,eq:mockup_system}.
Consider the \ac{mdp} induced by \(\rbM\).
Let \(\Uopt : \cX \times \cZ \to \bR\) be the value function for that \ac{mdp}.
Define \(Q : \cX \times \cZ \times \cA \to \bR\) by
\[
Q \of{x, z, a} = \onem{\delta} u \of{x, a} + \delta \expectcond[\rbM]{\Uopt \of{x\nxt, z\nxt}}{x, z, a},
\]
and \(\sigma\) by
\[
\sigma \of{x, z} \elmt{a} = \gibbsQ.
\]
As \(\tau\) goes to \(0\), \(\sigma\) converges to a \(\mum\)-optimal strategy.
When \(\tau\) is small enough, \(\sigma\) is \(\epsilonmum\) optimal.
To guarantee uniqueness, define \(\tau\) to be the largest value such that \(\sigma\) is \(\epsilonmum\) optimal.
Note that \(\sigma\) defined that way is a continuous function of the value function~\(\Uopt\).

One way to insure that~\cref{ass:ergodicity} is verified is to have a small noise affect all the transitions.
Formally, this means that for all \(w \in \cW\), \(x \in \cX\), \(a \in \cA\), and \(s \in \cS\), \(f \of{x, a, s}\), \(n \of{w, x, a}\), \(\nu \of{w}\), and \(\sigma \of{x, z}\) have full support.
From now on, \cref{ass:ergodicity} is always verified.

The following subsection extends the notion of \acp{eeo} to the multiagent case and defines \acp{eee}.

\subsection{Multiagent Setup}

Consider a collection of agents \(\cI\).
Each agent \(i\) has a state \(x\I\), an action \(a\I\), and a signal \(s\I\).
Let \(x\) be the tuple \(\tuple{\sseqcom{x}{\ag}{1}{2}{\abs{\cI}}}\).
Define \(a\) and \(s\) similarly.
Agent \(i\) is controlling the system described by
\begin{equation}
\label{eq:private_dynamic_i}
x\I\nxt \drawn  f\I \of{x\I, a\I, s\I}.
\end{equation}
Agents \(-i\) are controlling systems described as a whole by
\begin{equation}
\label{eq:private_dynamic_-i}
x\mI\nxt \drawn  f\mI \of{x\mI, a\mI, s\mI}.
\end{equation}
All these systems are coupled through Nature which determines the signals \(s\) according to
\begin{subequations}
\label{eq:nature_system_i}
\begin{align}
\label{eq:nature_dynamic_i}
w\nxt &\drawn n \of{w, x, a}, \\
\label{eq:nature_signal_i}
s &\drawn \nu \of{w}.
\end{align}
\end{subequations}

Denote by \(\rbN\I\) the system from agent \(i\)'s perspective.
In the single-agent setup, \(\rbN\) was composed of a known part \cref{eq:private_dynamic} and an unknown part \cref{eq:nature_system}.
Similarly, \(\rbN\I\) has a known part~\cref{eq:private_dynamic_i} and an unknown part~ \cref{eq:private_dynamic_-i,eq:nature_system_i}.

The other definitions from previous sections can readily be extended to the multiagent case.
Agent \(i\) has a utility function \(u\I\), a discount factor \(\delta\I\), a strategy \(\sigma\I : \cP\I \to \distribover{\cA\I}\), and a mockup of Nature and its opponents described by a state \(z\I\), a model \(m\I\), and a predictor \(\mu\I\).

From agent \(i\)'s perspective, everything is identical to the single-agent setup.
The notions of \(\mum\) optimality, \(\epsilonmum\) optimality, and \(\sigmam\) consistency can be replaced by \(\muImI\) optimality, \(\epsilonImuImI\) optimality, and \(\sigmamI\) consistency respectively.
Therefore, the definition of \ac{eeo} readily extends to the multiagent setting.
\begin{definition}
Let \(\sigma\), \(\mmu\), and \(\epsilon\) such that for all \(i\) in \(\cI\), \(\sigma\I\) is a strategy, \(\mImuI\) is a mockup, and \(\epsilon\I\) is a positive number.
The pair \(\sigmamu\) is an \(m\) \ac{eee} if the following two conditions hold for all \(i\) in \(\cI\):
\begin{enumerate}
\item Strategy \(\sigma\I\) is \(\muImI\) optimal.
\item Predictor \(\mu\I\) is \(\sigmamI\) consistent with \(\rbN\).
\end{enumerate}
The pair \(\sigmamu\) is an \(\epsilonm\) \ac{eee} if the following two conditions hold for all \(i\) in \(\cI\):
\begin{enumerate}
\item Strategy \(\sigma\I\) is a \(\muImI\) optimal.
\item Predictor \(\mu\) is \(\sigmamI\) consistent with \(\rbN\).
\end{enumerate}
\end{definition}

For a given \(m\) and \(\epsilon\) such that for all \(i\) in \(\cI\), \(\epsilon\I\) is a positive number, denote by \(\FO\) the optimization mapping from predictors to strategies and by \(\FM\) the modeling mapping from strategies to  predictors.
These mappings are defined by direct extension of their single agent counterparts.
Define~\(\FF\), a mapping from the space of predictors to itself, by~\(\FF = \FM \compo \FO\).

\section{Existence of Empirical-evidence Equilibria}


Fix models \(m\) and \(\epsilon\) such that for all \(i\) in \(\cI\), \(\epsilon\I\) is a positive number.

\begin{theorem}
\label{res:mme_existence}
There exists an \(\epsilonm\) \ac{eee}.
\end{theorem}

\begin{proof}
First, show that \(\FF\) has a fixed point.
The set of predictors is representable by a product of simplices.
Therefore, \(\FF\) is a mapping from a convex and compact set to itself.
By \cref{res:modeling_continuity,res:optimization_continuity}, \(\FO\) and \(\FM\) are continuous.
As the composition of two continuous functions, \(\FF\) is continuous.
By application of Brouwer's fixed-point theorem, \(\FF\) has a fixed point.

\Cref{res:fixed_point_EEE} therefore implies that an \(\epsilonm\) \ac{eee} exists.
\end{proof}

\begin{proposition}
\label{res:fixed_point_EEE}
Let \(\mu\opt\) be a fixed point of \(\FF\).
Define \(\sigma\opt\) by \(\sigma\opt = \FO \of{\mu\opt}\).
The pair~\(\tuple{\mu\opt, \sigma\opt}\) is an \(\epsilonm\) \ac{eee}.
\end{proposition}

\begin{proof}
By definition, strategy \(\sigma\opt\) is \(\epsilonImuoptImI\) optimal.
Note that
\[\FM \of{\sigma\opt} = \FM \compo \FO \of{\mu\opt} = \FF \of{\mu\opt} = \mu\opt.
\]
This implies that predictor \(\mu\opt\) is \(\sigmaoptmI\) consistent with~\(\rbN\I\).
Therefore, \(\tuple{\mu\opt, \sigma\opt}\) is an \(\epsilonm\)~\ac{eee}.
\end{proof}

\begin{proposition}
\label{res:optimization_continuity}
The optimization mapping \(\FO\) is continuous.
\end{proposition}

\begin{proof}
Agent \(i\)'s predictor only affects agent \(i\)'s strategy.
Therefore, proving that \(\FO\) is continuous, only requires showing that \(\FO\I: \mu\I \mapsto \sigma\I\) is continuous for all \(i \in \cI\).
Decomposing this function as follows:
\[
\FO\I: \mu\I \xmapsto{\subfunc{a}} U\I\opt \xmapsto{\subfunc{b}} \sigma\I,
\]
it is sufficient to prove that \(\subfunc{a}\) and \(\subfunc{b}\) are continuous.

\cref{res:markov_continuity} shows that the value function of a finite \ac{mdp} is a continuous function of the parameters of the problem.
Since \(\mu\I\) is one of the parameters of the \ac{mdp} whose value function is~\(U\I\opt\), \(\subfunc{a}\) is continuous.
It was noted in~\cref{sec:empirical_evidence_optimality} that \(\subfunc{b}\) is continuous.
\end{proof}

\begin{proposition}
\label{res:modeling_continuity}
The modeling mapping \(\FM\) is continuous.
\end{proposition}

\begin{proof}
Agent \(i\)'s strategy impacts all the agents' predictors.
Proving the continuity of \(\FM\), requires showing that \(\FM\ag{i, j}: \sigma\I \mapsto \mu\ag{j}\) is continuous for all \(i, j \in \cI\).
Decomposing this function as follows:
\[\FM\ag{i, j}: \sigma\I \xmapsto{\subfunc{c}} T\strat{\sigma} \xmapsto{\subfunc{d}} \pi\strat{\sigma} \xmapsto{\subfunc{e}} \mu\ag{j},
\]
it is sufficient to prove that \(\subfunc{c}\), \(\subfunc{d}\), and \(\subfunc{e}\) are continuous.

Since \(\subfunc{c}\) is linear, it is continuous.
\cite[Theorem~4.1]{meyer:1980} shows that the stationary distribution of a finite ergodic Markov chain is a continuous function of the elements of its transition matrix, which proves that \(\subfunc{d}\) is continuous.
It was noted in~\cref{sec:empirical_evidence_optimality} that \(\subfunc{e}\) is continuous.
\end{proof}

\begin{lemma}
\label{res:markov_continuity}
Consider a finite \ac{mdp} described by a dynamic~\(x\nxt \drawn f \of{x, a}\), a utility function~\(u \of{x, a}\), and a discount factor~\(\delta\).
Denote by \(\theta\) the finite vector of all the entries in \(f\) and~\(u\).
Let~\(\Btheta\) be the Bellman operator associated with the problem.
By definition, the value function of the problem \(\Utheta\) is the fixed point of \(\Btheta\), \(\Utheta = \Btheta \Utheta\).

The function \(\theta \mapsto \Utheta\) is continuous.
\end{lemma}

\begin{proof}
Let \(\theta\) and \(\theta'\) be two vectors of parameters.
The value function \(\Utheta\) is a fixed point of~\(\Btheta\). The Bellman operator~\(\Btheta\) is a contraction mapping with Lipschitz constant \(\delta\). As a result,
\begin{align*}
\norm{\Utheta - \Uthetap} &= \norm{\Btheta \Utheta - \Uthetap} \\
&\le \norm{\Btheta \Utheta - \Btheta \Uthetap} + \norm{\Btheta\Uthetap - \Uthetap} \\
&\le \delta \norm{\Utheta - \Uthetap} + \norm{\Btheta\Uthetap - \Uthetap} \\
&\le \frac{1}{\onem{\delta}} \norm{\Btheta\Uthetap - \Uthetap}.
\end{align*}

The continuity of \(\theta \mapsto \Btheta \Uthetap\) can now be established.
By definition, \(\parentheses{\Btheta \Uthetap} \of{x} = \max\idxin{a}{\cA} v \of{x, a, \theta}\), where \(v \of{x, a, \theta} = \onem{\delta} u \of{x, a} + \delta \tr{f \of{x, a}} \Uthetap\).
For fixed~\(x\) and~\(a\), \(\theta \mapsto v \of{x, a, \theta}\) is linear and therefore continuous.
For a fixed~\(x\), \(\theta \mapsto \Btheta \Uthetap \of{x}\) is the maximum of a finite number of continuous functions and as such is continuous.
The function \(\theta \mapsto \Btheta \Uthetap\) is continuous because each of its finitely many components is continuous.

Continuity of \(\theta \mapsto \Btheta \Uthetap\) implies that \(\norm{\Utheta - \Uthetap}\) goes to zero as \(\theta\) goes to \(\theta'\).
This last statement concludes the proof.
\end{proof}

\section{Learning Empirical-evidence Equilibria}
\label{sec:learning_empirical_evidence_equilibria}

\subsection{A Learning Rule}

The fixed points of \(\FF\) are \(\epsilonm\) \acp{eee}.
A natural approach to try and learn an \(\epsilonm\) \ac{eee} is to use an adaptive rule that converges to fixed points.
Consider the following adaptive rule:
\begin{equation}
\label{eq:update_rule}
\mu\Tp = \mu\T + \alpha\T \parentheses*{\FF \of{\mu\T} - \mu\T},
\end{equation}
where \(\alpha\T\) is a step size.
The long-run behavior of \cref{eq:update_rule} is related to properties of the following differential equation:
\begin{equation*}
\label{eq:differential_equation}
\dot{\mu} = \FF \of{\mu} - \mu.
\end{equation*}
In particular, Benaïm showed that the limit set of \cref{eq:update_rule} is a connected set internally chain-recurrent for the flow induced by \(\FF - \text{Id}\), where \(\text{Id}\) is the identity function~\cite{benaim:1996}.
The fixed points of \(\FF\) are connected sets internally chain-recurrent for the flow induced by~\(\FF - \text{Id}\) but they might not be the only ones.
Therefore, if \cref{eq:update_rule} converges it might yield an \(\epsilonm\)~\ac{eee}.

\subsection{Simulation Results}
This learning rule was successfully used on a simplified market example.
Two agents can hold a quantity of a single asset between \(0\) and \(4\), \(\cX = \set{0,1,2,3,4}\).
At each time step, each agent can sell one asset, buy one asset, or hold its position, \(\cA = \set{\text{Sell}, \text{Hold}, \text{Buy}}\).
The assets can be traded at a low price or at a high price, \(\cS=\set{\text{Low}, \text{High}}\).
Nature exogenously determines the market trend as a bull market or a bear market, \(\cW=\cS \times \set{\text{Bear}, \text{Bull}}\).
The price is impacted by the past price, the market trend, and the orders placed by the two agents.
A high price in the past, buying orders, or a bull market increase the chances of seeing a high price in the future.
The agents receive the price at each time step but are not aware of the price dynamic.
In this model, they are not even aware of the existence of the market trend.
The two agents use a discount factor~\(\delta=0.95\).

Agent~1 starts with the idea that the price will be high with probability 1.
Agent~2 starts with the idea that the price will be low with probability 1.
Each agent is trying to learn a depth-\(0\) model of the price.
Two versions of~\cref{eq:update_rule} were simulated.
The first one used~\cref{eq:update_rule} directly with a fixed step size of~\(0.1\).
The stationary distribution \(\pi\strat{\sigma}\) was computed at each time step to obtain the true value of~\(\FF \of{\mu\T}\),
The results of the simulations using the theoretical predictor are presented in~\cref{fig:simulation_theoretical}.
Since the price is a public signal, after a transient phase due to the step size, the predictions of both agents agree.
The prediction converges to probability of seeing a high price of \(0.431\).
The two agents use the same strategy that is the optimal response for that prediction of the price.
When the price is high sell.
When the price is low, sell when having four units, hold when having three units, and buy otherwise.
The learning rule has indeed converged to an \ac{eee}.

\newcommand{\figureswidth}{320pt}
\newcommand{\figuresheight}{170pt}

\begin{figure}[!ht]
  \centering
  \newcommand{\ylabelsymbol}{\(\mu\T\I\elmt{\mathrm{High}}\)}%
  \newcommand{\tabledata}{simulation_theoretical.dat}%
  \begin{tikzpicture}
    \pgfplotstableread{\tabledata}{\simulationdata}
\begin{axis}[
width=\figureswidth,
height=\figuresheight,
xlabel=Time \(t\),
ylabel=Prediction \ylabelsymbol,
ymin=0,
ymax=1,
xmin=0,
xmax=100,
legend entries={\(i = 1\), \(i = 2\)}
],
\addplot[thick, solid] table [x={time}, y={phigh1}] {\simulationdata};
\addplot[thick, dashed] table [x={time}, y={phigh2}] {\simulationdata};
\end{axis}

  \end{tikzpicture}
  \caption{Simulation results of two agents learning a depth-0 model of the price for the market example with the theoretical predictor.}
  \label{fig:simulation_theoretical}
\end{figure}

\newpage
In the second simulated version of~\cref{eq:update_rule}, the stationary distribution was only estimated by playing~\(100\) rounds of the game at each time step.
Because of the variance induced by this sampling process, the step size was taken to be diminishing, \(\alpha\T = \parentheses*{\frac{1}{t}}\pow{\frac{3}{4}}\).
The estimated predictors obtained in that case are denoted by \(\hat{\mu}\I\T\).
The results of the simulation using the empirical predictors are presented in~\cref{fig:simulation_empirical}.
Estimating, instead of using the true probability, induces some variations.
The learning rule does not converge, but oscillates around the \ac{eee} reached by the theoretical predictor.

\begin{figure}[!ht]
  \centering
  \newcommand{\ylabelsymbol}{\(\hat{\mu}\T\I\elmt{\mathrm{High}}\)}%
  \newcommand{\tabledata}{simulation_empirical.dat}%
  \begin{tikzpicture}
    \pgfplotstableread{\tabledata}{\simulationdata}
\begin{axis}[
width=\figureswidth,
height=\figuresheight,
xlabel=Time \(t\),
ylabel=Prediction \ylabelsymbol,
ymin=0,
ymax=1,
xmin=0,
xmax=100,
legend entries={\(i = 1\), \(i = 2\)}
],
\addplot[thick, solid] table [x={time}, y={phigh1}] {\simulationdata};
\addplot[thick, dashed] table [x={time}, y={phigh2}] {\simulationdata};
\end{axis}

  \end{tikzpicture}
  \caption{Simulation results of two agents learning a depth-0 model of the price for the market example with an empirical predictor.}
  \label{fig:simulation_empirical}
\end{figure}
