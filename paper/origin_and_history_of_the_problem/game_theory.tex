\subsection{Games and Equilibria}
In a game setting, a set of agents \(\cI\) faces decision-making problems.
Each agent \(i\) in \(\cI\) has an action set \(\cA\I\) and a utility function \(u\I \from \cA \to \bR\), where \(\cA = \prod \idxin{i}{\cI} \cA\I\) is called the joint action set.
Note that this utility function depends on the actions of all the agents.
As mentioned earlier, decision making is the rational process of finding an optimal action given the information available.
There is no obvious way to extend that definition to the multiagent setting.
Preferences of different agents cannot be aggregated; therefore, the notion of optimality for the set of agents is ill defined.

Optimality for an individual agent is still well defined.
Denote the opponents of agent~\(i\) by \(-i = \cI \setminus \set{i}\).
For fixed actions of its opponents, agent~\(i\) faces a decision-making problem.
The actions in \(\cA\I\) that are optimal for the fixed actions of \(-i\) are called best responses.
For \(a\mI = \tuple{\sseqcom{a}{\ag}{1}{2}{i-1},\seqcom{a}{\ag}{i+1}{\card{\cI}}}\), the best-response set of agent \(i\) is~\(\br\I \of {a\mI} = \argmax\idxin{a\I}{\cA\I} u\I \of {a\I, a\mI}\).
Note that \(\br\I \from \cA\mI \tto \cA\I\) is a correspondence and not a function.

A joint action that is simultaneously a best response for all the agents is a good candidate to replace optimality in the multiagent setting.
This concept, at the core of game theory, is called a Nash equilibrium.
A joint action \(a\opt\) in \(\cA\) is a Nash equilibrium if
\begin{equation}
\label{eq:nash_equilibrium}
\forall i \in \cI \comma a\opt\I \in \br\I \of {a\opt\mI}.
\end{equation}
The definitions of best response and Nash equilibrium extend readily to mixed actions by replacing \(a\)s by \(\alpha\)s.
Note that a deterministic action is sometimes called a pure action.
Nash proved that any game with a finite number of players having a finite number of actions has at least one, potentially mixed, Nash equilibrium~\cite{nash:1951}.

\subsection{A Game Example}
Game-theoretic concepts are illustrated on the the following game known as battle of the sexes.
A couple, composed of a man \(\rlM\) and a woman \(\rlW\), is planning a date.
Each one chooses between two actions: going to a football match \(\rF\) or going to an opera performance \(\rO\).
The joint action of the couple is represented by an ordered pair \(\tuple{a\ag{\rlM}, a\ag{\rlW}}\), where \(a\ag{\rlM}\) is the action chosen by the man and \(a\ag{\rlW}\) by the woman.
For example, \(\tuple{\rF, \rO}\) denotes that he chooses football and she chooses opera.

The man prefers to be with the woman rather than separated from her.
If they are together, he prefers football \(\tuple{\rbF, \rF}\) to opera \(\tuple{\rbO, \rO}\).
If they are not together, he is indifferent between football \(\tuple{\rbF, \rO}\) and opera \(\tuple{\rbO, \rF}\).
The woman prefers to be with the man rather than separated from him.
If they are together, she prefers opera \(\tuple{\rO, \rbO}\) to football \(\tuple{\rF, \rbF}\).
If they are not together, she still prefers opera \(\tuple{\rF, \rbO}\) to football \(\tuple{\rO, \rbF}\).
Their preferences can be implemented by utility functions \(u\ag{\rlM}\) for the man and \(u\ag{\rlW}\) for the woman with the following values:
\begin{equation}
\begin{aligned}
u\ag{\rlM} \of {\rF, \rF}  &= 2, & u\ag{\rlM} \of {\rO, \rO}  &= 1, & u\ag{\rlM} \of {\rF, \rO}  &= 0, & u\ag{\rlM} \of {\rO, \rF}  &= 0, \\
u\ag{\rlW} \of {\rO, \rO}  &= 3, & u\ag{\rlW} \of {\rF, \rF}  &= 2, & u\ag{\rlW} \of {\rF, \rO}  &= 1, & u\ag{\rlW} \of {\rO, \rF}  &= 0.
\end{aligned}
\end{equation}

The action sets and the utility functions of battle of the sexes are represented in a compact form as follows:
\begin{equation}
\label{eq:bos_normal_form}
\punctuategame{
\begin{game}{2}{2}
        \> \(\rF\) \> \(\rO\)  \\
\(\rF\) \> 2, 2    \> 0, 1 \\
\(\rO\) \> 0, 0    \> 1, 3
\end{game}}
{.}\bigskip
\end{equation}
The man's action determines the row and the woman's determines the column.
Numbers in the cell are the utilities received: the first by the man and the second by the woman.
This is called the normal-form representation of the game.

To compute the best response of the man, fix the action \(\alpha\ag{\rlW}\) of the woman.
With~\(\alpha\ag{\rlW}\), she chooses \(\rF\) with probability \(p\ag{\rlW}\) and \(\rO\) with probability \(\onem{p\ag{\rlW}}\).
Note that she chooses a pure action for \(p\ag{\rlW}=0\) or \(p\ag{\rlW}=1\).
The utility received by the man, if he plays \(\rF\), is~\(u\ag{\rlM} \of {\rF, \alpha\ag{\rlW}} = p\ag{\rlW} u\ag{\rlM} \of {\rF, \rF} + \onem{p\ag{\rlW}} u\ag{\rlM} \of {\rF, \rO} = 2p\ag{\rlW}\).
If he plays \(\rO\), his utility is \(u\ag{\rlM} \of {\rO, \alpha\ag{\rlW}} = p\ag{\rlW} u\ag{\rlM} \of {\rO, \rF} + \onem{p\ag{\rlW}} u\ag{\rlM} \of {\rO, \rO} = 1 - p\ag{\rlW}\).
His optimal action depends on \(p\ag{\rlW}\) with a critical value of \(\frac{1}{3}\).
If~\(p\ag{\rlW} > \frac{1}{3}\), he prefers \(\rF\) to \(\rO\).
If~\(p\ag{\rlW} < \frac{1}{3}\), he prefers \(\rO\) to \(\rF\).
If~\(p\ag{\rlW} = \frac{1}{3}\), he is indifferent between \(\rF\) and \(\rO\); any combination of \(\rF\) and \(\rO\) is a best response.


The best response of the woman is computed in a similar fashion.
The critical value of \(p\ag{\rlM}\) making her indifferent is \(\frac{3}{4}\).

The best responses are plotted in~\cref{fig:bos_best_response}.
The intersections of the graphs correspond to the Nash equilibria of the game.
Battle of the sexes has three Nash equilibria: two pure ones and one mixed.
The pure Nash equilibria arise from the man and the woman choosing the same event.
The mixed one corresponds to the man and the woman independently randomizing their choices with probabilities \(p\ag{\rlM} = \frac{3}{4}\) and \(p\ag{\rlW} = \frac{1}{3}\).

\begin{figure}[!ht]
\caption[Best responses and Nash equilibria for battle of the sexes.]{
Best responses and Nash equilibria for battle of the sexes.
The man plays~\(\rF\) with probability~\(p\ag{\rlM}\).
The woman plays~\(\rF\) with probability~\(p\ag{\rlW}\).
The solid line is the man's best response.
The dashed line is the woman's best response.
The filled circles indicate the Nash equilibria.
}
\label{fig:bos_best_response}
\centering
\begin{tikzpicture}
\pgfplotsset{bestresponse/.style={mark=none}}
\pgfplotsset{nashequilibrium/.style={only marks}}

\begin{axis}[
axis lines=middle,
clip=false,
xmax=1.1,
ymax=1.1,
height=193pt,
width=193pt,
ticklabel style={/pgf/number format/.cd,frac},
xtick={0.75, 1},
ytick={0.3333333, 1},
xlabel=\(p\ag{\rlM}\),
xlabel style={anchor=north},
ylabel=\(p\ag{\rlW}\),
ylabel style={anchor=east},
],
\addplot[bestresponse] coordinates {(0,0) (0,1/3) (1,1/3) (1,1)};
\node[anchor=west] at (axis cs:1,0.666666) {\(\br\ag{\rlM}\)};

\addplot[bestresponse, dashed] coordinates {(0,0) (3/4,0) (3/4,1) (1,1)};
\node[anchor=east] at (axis cs:0.75,0.666666) {\(\br\ag{\rlW}\)};

\addplot[nashequilibrium] coordinates {(0,0) (3/4,1/3) (1,1)};
\node[anchor=north east] at (axis cs:0,0) {\(0\)};
\end{axis}

\end{tikzpicture}
\end{figure}

\subsection{Are Nash Equilibria the Solution?}
\label{sec:are_nash_equilibria_the_solution}
Nash equilibria form a solution concept for multiagent decision problems.
Nash equilibria are self-enforcing agreements; if you give actions forming a Nash equilibrium to some agents, none of them has an incentive to unilaterally deviate.
This self-enforcing property makes Nash equilibria relevant for decentralized control.
Nash equilibria cannot, however, be called \textbf{the} solution to multiagent decision problems.
They have weaknesses made clear by the following questions:
\begin{itemize}
\item When a game has multiple equilibria, how do agents coordinate to choose one equilibrium?
\item Different utility functions can represent the same preferences.
However, changing the utility functions changes the probabilities in a mixed Nash equilibrium.
Therefore, what is the intrinsic meaning of a mixed Nash equilibrium?
\item The social welfare is the sum of the utilities of the agents.
Should the social welfare be maximized instead of finding an equilibrium?
\item A Nash equilibrium guarantees no profitable unilateral deviation.
What happens when two agents deviate simultaneously?
\end{itemize}
Authors have been debating for decades regarding which equilibrium is the right solution concept.
This debate is not relevant for the proposed research.
Any solution concept can be used as long as its limitations are understood.

\subsection{Correlated Equilibria}
\label{sec:correlated_equilibria}
The mixed Nash equilibrium in battle of the sexes seems fairer than the two pure ones: each agent has a chance to go on his or her preferred date.
However, the two agents end up in different locations with positive probability: some utility is wasted.
In the mixed Nash equilibrium, the expected utility for the man is \(\frac{2}{3}\) and for the woman \(\frac{3}{2}\).

When facing these kinds of incompatible decisions, humans sometimes have recourse to a coin toss.
The agents agree that on heads they go to the football match and on tails they go to the opera performance.
This is not a Nash equilibrium.
However, once the outcome of the coin toss is known, no agent has an incentive to unilaterally deviate.
This is another form of equilibrium, introduced by Aumann under the name correlated equilibrium~\cite{aumann:1987}.
The correlated equilibrium maintains the fairness of the mixed Nash equilibrium but does not waste utility.
In the correlated equilibrium, the expected utility for the man is \(\frac{3}{2}\) and for the woman \(\frac{5}{2}\).
Note that every Nash equilibrium is a correlated equilibrium but the converse is not true.

Two equivalent definitions of a correlated equilibrium exist.
The first one describes the equilibrium in terms of the common random source and the actions taken given each possible outcome; the second one, in terms of the resulting distribution over joint actions.
The first definition is more intuitive whereas the second one is easier to work with.
As a result, we give the second definition.
A probability distribution over joint actions \(\gamma\) is a correlated equilibrium if
\begin{equation}
\label{eq:correlated_equilibrium}
\forall i \in \cI \comma a\I, a'\I \in\cA\I \comma \sum\idxin{a\mI}{\cA\mI} \gamma\elmt{a} \grpparen{u\I \of {a} - u\I \of {a'\I, a\mI}} \geq 0.
\end{equation}
