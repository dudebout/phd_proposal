\section*{Symbols}

\(t\) denotes a discrete time step.
\(b\T\) is the value of variable \(b\) at time \(t\).
When unambiguous, \(b\) and \(b\nxt\) are short notations for \(b\T\) and \(b\Tp\) respectively.

\(\distribover{\cB}\) is the set of distributions over finite set \(\cB\).
\(b \drawn \beta\) denotes that \(b\) is drawn according to distribution \(\beta\).
\(\probaof[\beta]{B}\) is the probability of event \(B\) under distribution \(\beta\).
\(\beta \elmt {e}\) denotes the quantity \(\probaof[\beta]{b = e}\), for \(b \drawn \beta\).
\(\expectof[\beta]{b}\) is the expected value of \(b\) under distribution \(\beta\).

\(\cI\) is a set of agents and \(i\) denotes one agent.
\(-i\) represents the set of all agents excluding agent \(i\), \ie, \(\cI \setminus \set{i}\).
If \(\cB\I\) is a set associated with agent \(i\), \(\cB\) denotes the Cartesian product \(\prod\idxin{i}{\cI} \cB\I\).
If \(b\I\) is a variable associated with agent \(i\), \(b\) denotes the tuple \(\tuple{\sseqcom{b}{\ag}{1}{2}{\card{\cI}}}\).

\printglossary[type=acronym,title=Acronyms\bigskip]
